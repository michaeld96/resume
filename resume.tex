\documentclass[11pt]{article}
\usepackage{graphicx}

\usepackage[a4paper, total={7.8in, 10.8in}, top = 0.5in]{geometry}

% Hypertext package for font and functionality.
\usepackage{hyperref}
\usepackage{lipsum}
% This turns off the page numbering.
\pagenumbering{gobble} 

\hypersetup{
    colorlinks=true,
    urlcolor = blue
}

% No auto indent.
\setlength\parindent{0pt}

\usepackage{calc}
\newlength{\remaining}

\begin{document}

\scalebox{1.75}{\textbf{Michael Dick}} \hfill E-mail: {\color{black}\url{ mikedick@umich.edu}}
\par Website: \url{http://miketdick.com/} \hfill Phone: 616-902-8922
\par GitHub: \url{https://github.com/michaeld96}
\newline
%%%%%%%%%%%%%%%%%%%%%%%%%%%%%%%%%%% Section Start %%%%%%%%%%%%%%%%%%%%%%%%%%%%%%%%%%%
\par \scalebox{1.3}{E}DUCATION                    
\par \vspace{-0.1in} \noindent\rule{7.8in}{0.5pt} 
%%%%%%%%%%%%%%%%%%%%%%%%%%%%%%%%%%% Section End   %%%%%%%%%%%%%%%%%%%%%%%%%%%%%%%%%%%
\textbf{\scalebox{1.2}{University of Michigan - Ann Arbor}}
\par \textit{Bachelor of Science in Engineering - Computer Science, Mathematics Minor} \hfill \textit{Expected: May 2024}
{\small 

\begin{itemize}
    \item GPA: 3.62/4.00
    \item \textbf{Relevant Coursework:} EECS 498 (Game Engine Architecture), EECS 442 (Computer Vision), EECS 471 (GPU Programming), EECS 483 (Compilers), EECS 281 (Data Structures and Algorithms), EECS 370 (Computer Architecture), ROB 102 (Intro to AI programming), MATH 425 (Probability), MATH 423 (Mathematical Finance)
    \item \textbf{Organizations and Activities:} IT Council Student Representative, First Generation Engineers, Michigan Student Artificial Intelligence Lab, UROP (Undergraduate Research Opportunity) 
\end{itemize}
}

%%%%%%%%%%%%%%%%%%%%%%%%%%%%%%%%%%% Section Start %%%%%%%%%%%%%%%%%%%%%%%%%%%%%%%%%%%
\par \scalebox{1.3}{E}XPERIENCE                    
\par \vspace{-0.1in} \noindent\rule{7.8in}{0.5pt} 
%%%%%%%%%%%%%%%%%%%%%%%%%%%%%%%%%%% Section End   %%%%%%%%%%%%%%%%%%%%%%%%%%%%%%%%%%%
\textbf{\scalebox{1.2}{Amador Bioscience}} \hfill \scalebox{1.1}{Ann Arbor, Michigan}
\par \textit{Software Engineer} \hfill \textit{January 2023 - Present}
{\small
\begin{itemize}
    \item Developed and launched 'APMX', an open-source R package designed to simplify data cleaning and formatting for PK/PD analysis, making it easier to use in NONMEM. This tool was presented at the PAGE conference in Spain.
    \item Implemented a testing suite using automated unit tests to ensure our code was consistently accurate and reliable. This included checking calculations and formatting with snapshots of processed data, which greatly improved the quality and dependability of our software.
    \item Took on a key role in coding and enhancing various functions within APMX, focusing on user-friendly features and efficient data processing. My contributions helped make complex data sets more manageable and interpretable for users.
\end{itemize}
}

\textbf{\scalebox{1.2}{Ann Arbor Pharmacometrics Group}} \hfill \scalebox{1.1}{Ann Arbor, Michigan}
\par \textit{Software Engineer} \hfill \textit{May 2022 - December 2022}
{
\small
\begin{itemize}
    \item Developed a Noncompartmental Analysis tool to assist pharmacometricians in generating exploratory plots, tables, listings, and figures. The tool facilitates easy data input and editing, allowing users to eliminate unwanted outliers. Built using R, R-Shiny, HTML, SASS, and JavaScript.
    \item Containerized the developed application using Docker, enhancing accessibility across various operating systems and environments, thereby making it widely available to users.
    \item Deployed the containerized application on Amazon Web Services using the Elastic Container Service, ensuring accurate configuration of environments and execution paths for reliable R code operation and application launch.
    \item Streamlined the graph rendering process, significantly reducing the rendering time from 5 seconds to 1 second. Achieved this optimization by implementing efficient caching methods to accelerate the generation of a series of graphs.
\end{itemize}
}

\textbf{\scalebox{1.2}{Rackham Graduate School - University of Michigan}} \hfill \scalebox{1.1}{Ann Arbor, Michigan}
\par \textit{Undergraduate Researcher} \hfill \textit{October 2019 - September 2020}
{\small
\begin{itemize}
    \item Utilized Adobe Animate, Audacity, Adobe Premiere Rush, and Adobe Illustrator to create an educational animation. The project was based on the research paper: “The Mentor’s Dilemma: Providing Critical Feedback Across the Racial Divide” by G. Cohen, C. Steele, and L. Ross.
    \item The animation, aimed at incoming graduate students, highlights the importance of feedback in academic settings and its intersection with racial dynamics.
    \item Collaborated closely with Professor Adam J. Matzger, who guided the project, to explore effective presentation mediums and strategies for disseminating research findings.
\end{itemize}
}

% \textbf{\scalebox{1.2}{Two Men and a Truck}} \hfill \scalebox{1.1}{Ann Arbor, Michigan}
% \par \textit{Logistics and Sales Intern} \hfill \textit{May 2019 - September 2019}
% {\small
% \begin{itemize}
%     \item Role was responsible for 90\% of the revenue for our franchise. Was number one in revenue out of all our interns in our district. Total income generated while employed for this role was a total over \$130,000. 
%     \item Would organize routes and dispatch movers and drivers to job locations.
%     \item Took inbound calls and requests via email from clients, planned clients' request for a move, and was responsible for the clients' satisfaction with the services provided. 
% \end{itemize}
% }
%%%%%%%%%%%%%%%%%%%%%%%%%%%%%%%%%%% Section Start %%%%%%%%%%%%%%%%%%%%%%%%%%%%%%%%%%%
\par \scalebox{1.3}{S}KILLS \scalebox{1.3}{S}UMMARY                  
\par \vspace{-0.1in} \noindent\rule{7.8in}{0.5pt} 
%%%%%%%%%%%%%%%%%%%%%%%%%%%%%%%%%%% Section End   %%%%%%%%%%%%%%%%%%%%%%%%%%%%%%%%%%%

\begin{itemize}
    \item \textbf{Languages:} \hspace*{3cm} C/C++, CUDA, Rust, Python, R, JavaScript, HTML, CSS, SQL, C\#
    \item \textbf{Tools:} \hspace*{4cm} Git, \LaTeX, GNU Makefile, AWS (Amazon Web Services), Docker
    % \item \textbf{Environments:} \hspace*{2.35cm} VIM, VS Code, Visual Studio, Xcode, Rstudio, *nix Systems
    \item \textbf{Frameworks/Libraries:} \hspace*{0.9cm}SDL/SDL2, ASP.NET, React, OpenGL, R-Shiny, Numpy, OpenCV
\end{itemize}

\end{document}