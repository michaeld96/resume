\documentclass[11pt]{article}
\usepackage{graphicx}

\usepackage[a4paper, total={7.8in, 10.8in}, top = 0.5in]{geometry}

% Hypertext package for font and functionality.
\usepackage{hyperref}
\usepackage{lipsum}
% This turns off the page numbering.
\pagenumbering{gobble} 

\hypersetup{
    colorlinks=true,
    urlcolor = blue
}

% No auto indent.
\setlength\parindent{0pt}

\usepackage{calc}
\newlength{\remaining}

\begin{document}

\scalebox{1.75}{\textbf{Michael Dick}} \hfill E-mail: {\color{black}\url{ mikedick@umich.edu}}
\par Website: \url{http://miketdick.com/} \hfill Phone: 616-902-8922
\par GitHub: \url{https://github.com/michaeld96}
\vspace*{0.1cm}
% \newline
%%%%%%%%%%%%%%%%%%%%%%%%%%%%%%%%%%% Section Start %%%%%%%%%%%%%%%%%%%%%%%%%%%%%%%%%%%
\par \scalebox{1.3}{E}DUCATION                    
\par \vspace{-0.1in} \noindent\rule{7.8in}{0.5pt} 
%%%%%%%%%%%%%%%%%%%%%%%%%%%%%%%%%%% Section End   %%%%%%%%%%%%%%%%%%%%%%%%%%%%%%%%%%%
\textbf{\scalebox{1.2}{University of Michigan - Ann Arbor}}
\par \textit{Bachelor of Science in Engineering - Computer Science, Mathematics Minor} \hfill \textit{Expected: May 2024}
{\small 

\begin{itemize}
    \item GPA: 3.62/4.00
    \item \textbf{Relevant Coursework:} Applied Parallel Programming with GPUs, Computer Vision, Compilers, Game Engine Architecture, 
    Human-Centered Software Design and Development, Data Structures and Algorithms, Intro to Computer Architecture
    % \item \textbf{Organizations and Activities:} IT Council Student Representative, First Generation Engineers, Michigan Student Artificial Intelligence Lab, UROP (Undergraduate Research Opportunity) 
\end{itemize}
}

%%%%%%%%%%%%%%%%%%%%%%%%%%%%%%%%%%% Section Start %%%%%%%%%%%%%%%%%%%%%%%%%%%%%%%%%%%
\par \scalebox{1.3}{E}XPERIENCE                    
\par \vspace{-0.1in} \noindent\rule{7.8in}{0.5pt} 
%%%%%%%%%%%%%%%%%%%%%%%%%%%%%%%%%%% Section End   %%%%%%%%%%%%%%%%%%%%%%%%%%%%%%%%%%%
\textbf{\scalebox{1.2}{Amador Bioscience}} \hfill \scalebox{1.1}{Ann Arbor, Michigan}
\par \textit{Software Engineer} \hfill \textit{January 2023 - Present}
{\small
\begin{itemize}
    \item Developed and launched `APMX', an open-source R package designed to simplify data cleaning and formatting for PK/PD analysis, making it easier to use in NONMEM. This tool was presented at the PAGE conference in Spain.
    \item Implemented a testing suite using automated unit tests to ensure our code was consistently accurate and reliable. This included checking calculations and formatting with snapshots of processed data, which greatly improved the quality and dependability of our software.
    \item Took on a key role in coding and enhancing various functions within APMX, focusing on user-friendly features and efficient data processing. My contributions helped make complex data sets more manageable and interpretable for users.
\end{itemize}
}

\textbf{\scalebox{1.2}{Ann Arbor Pharmacometrics Group}} \hfill \scalebox{1.1}{Ann Arbor, Michigan}
\par \textit{Software Engineer} \hfill \textit{May 2022 - December 2022}
{
\small
\begin{itemize}
    \item Developed a Noncompartmental Analysis tool to assist pharmacometricians in generating exploratory plots, tables, listings, and figures. The tool facilitates easy data input and editing, allowing users to eliminate unwanted outliers. Built using R, R-Shiny, HTML, SASS, and JavaScript.
    \item Containerized the developed application using Docker, enhancing accessibility across various operating systems and environments, thereby making it widely available to users.
    \item Deployed the containerized application on Amazon Web Services using the Elastic Container Service, ensuring accurate configuration of environments and execution paths for reliable R code operation and application launch.
\end{itemize}
}

\textbf{\scalebox{1.2}{Rackham Graduate School - University of Michigan}} \hfill \scalebox{1.1}{Ann Arbor, Michigan}
\par \textit{Undergraduate Researcher} \hfill \textit{October 2019 - September 2020}
{\small
\begin{itemize}
    \item Utilized Adobe Animate, Audacity, Adobe Premiere Rush, and Adobe Illustrator to create an educational animation. The project was based on the research paper: “The Mentor’s Dilemma: Providing Critical Feedback Across the Racial Divide” by G. Cohen, C. Steele, and L. Ross.
    \item The animation, aimed at incoming graduate students, highlights the importance of feedback in academic settings and its intersection with racial dynamics.
    \item Collaborated closely with Professor Adam J. Matzger, who guided the project, to explore effective presentation mediums and strategies for disseminating research findings.
\end{itemize}
}

%%%%%%%%%%%%%%%%%%%%%%%%%%%%%%%%%%% Section Start %%%%%%%%%%%%%%%%%%%%%%%%%%%%%%%%%%%
\par \scalebox{1.3}{P}ROJECTS                  
\par \vspace{-0.1in} \noindent\rule{7.8in}{0.5pt} 
%%%%%%%%%%%%%%%%%%%%%%%%%%%%%%%%%%% Section End   %%%%%%%%%%%%%%%%%%%%%%%%%%%%%%%%%%%

\textbf{Ray Tracer:} Built a ray tracer from the scratch. Ray tracer would write to PPM files, which could then be converted to PNGs. Implemented features such as shadows, reflections, refractions, and anti-aliasing. \hfill \textit{C++}
\newline
\textbf{Game Engine:} Developed a game engine from the ground up. Engine supports 2D, audio, input, and physics. Engine also supports scripting using Lua. \hfill 
\textit{C++, Lua, SDL2}
\newline
\textbf{Rust Compiler:} Implemented a compiler for a custom language named Snake. Compiler supports basic arithmetic, conditionals, loops, functions, autonomous functions, arrays, and floating point operations. \hfill \textit{Rust}
\newline
\textbf{Reverse Image Search:} Fine-tunded EfficientNet B0 to 90\% accuracy on a set of 5000 images to create high quality embeddings. Created a web application that allows users to upload images and find similar images. Utilized Postgres PGVector for efficient vector queries to retrieve most similar images. \hfill \textit{Python, PyTorch, Flask, React, VectorDB, SQL}
\newline

%%%%%%%%%%%%%%%%%%%%%%%%%%%%%%%%%%% Section Start %%%%%%%%%%%%%%%%%%%%%%%%%%%%%%%%%%%
\par \scalebox{1.3}{S}KILLS \scalebox{1.3}{S}UMMARY                  
\par \vspace{-0.1in} \noindent\rule{7.8in}{0.5pt} 
%%%%%%%%%%%%%%%%%%%%%%%%%%%%%%%%%%% Section End   %%%%%%%%%%%%%%%%%%%%%%%%%%%%%%%%%%%

\begin{itemize}
    \item \textbf{Languages:} \hspace*{3cm} C/C++, CUDA, Rust, Python, R, JavaScript, HTML, CSS, SQL, C\#, Java
    \item \textbf{Tools:} \hspace*{4cm} Git, \LaTeX, GNU Makefile, AWS, Docker
    % \item \textbf{Environments:} \hspace*{2.35cm} VIM, VS Code, Visual Studio, Xcode, Rstudio, *nix Systems
    \item \textbf{Frameworks/Libraries:} \hspace*{0.9cm}SDL/SDL2, ASP.NET, React, OpenGL, R-Shiny, Numpy, OpenCV
\end{itemize}

\end{document}